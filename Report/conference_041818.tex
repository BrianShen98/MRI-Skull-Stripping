\documentclass[conference]{IEEEtran}
\IEEEoverridecommandlockouts
% The preceding line is only needed to identify funding in the first footnote. If that is unneeded, please comment it out.
\usepackage{cite}
\usepackage{amsmath,amssymb,amsfonts}
\usepackage{algorithmic}
\usepackage{graphicx}
\usepackage{textcomp}
\usepackage{xcolor}
\def\BibTeX{{\rm B\kern-.05em{\sc i\kern-.025em b}\kern-.08em
    T\kern-.1667em\lower.7ex\hbox{E}\kern-.125emX}}
\begin{document}

\title{Skull Stripping for MRI: a deep neural network approach}

\author{\IEEEauthorblockN{Kaiyuan Chen}
\IEEEauthorblockA{\textit{Computer Science Depart.} \\
\textit{University of California, Los Angeles}\\
Los Angeles, United State \\
chenkaiyuan@ucla.edu}
\and
\IEEEauthorblockN{2\textsuperscript{nd} Given Name Surname}
\IEEEauthorblockA{\textit{dept. name of organization (of Aff.)} \\
\textit{name of organization (of Aff.)}\\
City, Country \\
email address}
}

\maketitle

\begin{abstract}
In this project, we developed a deep Convolutional Neural Network(CNN) scheme to perform brain skull stripping on Magnetic Resonance Image(MRI). We analyzed previous works on machine learning including ensembled learning and linear models. By conducting series of experiments, we find weaknesses of popular machine models on scalability and strong assumption on structure of images. In order to reduce above problems, we propose a CNN approach to reach a higher scalablity and visual accuracy. 

\end{abstract}

\begin{IEEEkeywords}
Medical Imaging, Machine Learning, Skull Stripping, MRI
\end{IEEEkeywords}

\section{Introduction}
Computer aided diagnosis based on medical images from MRI(magnetic resonance image) has gained ubiquitous usage for its “noninvasive, nondestructive, flexible” properties[2]. With the help of FLAIR(Fluid-attenuated inversion recovery), Diffusion-weighted(DW) MRI, people can get an anatomical structure of human soft tissues with high resolution. Especially to satisfy the demand for interior and exterior structure of brain structures, MRI can produce cross-sectional images from different angles, for example, top-down, side-to-side and front-to-back: however, having slices from different angles give a lot of challenges in stripping those tissues which people are interested in, from xtra-cranial or non-brain tissues that has nothing to do with brain diseases such as Alzheimer’s disease, aneurysm in the brain, arteriovenous malformation-cerebral and Cushings disease and etc[1]. 

As a preliminary step for further analysis, brain segmentation, i.e. skull stripping, needs both speed and accuracy in practice, which should be considered in any algorithms proposed. By Kalavathi et al. [2], they can be classified into five categories: mathematical morphology-based methods, intensity-based methods, deformable surface-based methods, atlas-based methods, and hybrid methods. However, as we further reviewed on state-of-arts that are vaguely described in hybrid methods, we believe machine learning-based methods should also have its own place in brain segmentation. Machine learning is a broad concept that include many interesting algorithms that we would like to implement and experiment on. For example, Butman introduced a robust machine learning method that detects the brain boundary by random forest[2]. As random forest has high expressive power on voxels of brain boundary, this method can reach an high accuracy robustly. Popular methods like deep learning can also applied. For example, Kleesiek et al.[8] used non-parametric 3D Convolutional Neural Network(CNN) to learn important features and reach the highest Dice score among all the methods we have reviewed. However, as a parametric algorithm, GMM also has its place in brain segmentation. For example, Yunjie et al. developed a skull stripping method with an adaptive gauss mixture model and a 3D mathematical morphology method. The GMM is used to classify brain tissues and to estimate the bias field in the brain tissues [5]. These methods, along with well-implemented libraries such as sklearn[6], Tensorflow[7], are readily available for our use.

Our contribution in this work is as following:
\begin{itemize}
\item We conducted a series of experiments on previous works of machine learning based skull stripping, which are based ensembled learning like random forest and linear models like SVM. 
\item To solve problems of previous works, we adopt CNN in a scheme similar to autoencoder. 
\item We manually labelled a lot of MRI images 
\end{itemize}

\section{Baseline Model}

\section{CNN model}

\section{Discussion and Future Work}
\begin{thebibliography}{00}
\bibitem{b1} G. Eason, B. Noble, and I. N. Sneddon, ``On certain integrals of Lipschitz-Hankel type involving products of Bessel functions,'' Phil. Trans. Roy. Soc. London, vol. A247, pp. 529--551, April 1955.

\end{thebibliography}
\end{document}
